%********************************************%
%*       Generated from PreTeXt source      *%
%*       on 2020-08-14T15:51:36-07:00       *%
%*   A recent stable commit (2020-08-09):   *%
%* 98f21740783f166a773df4dc83cab5293ab63a4a *%
%*                                          *%
%*         https://pretextbook.org          *%
%*                                          *%
%********************************************%
\documentclass[oneside,10pt,]{article}
%% Custom Preamble Entries, early (use latex.preamble.early)
%% Default LaTeX packages
%%   1.  always employed (or nearly so) for some purpose, or
%%   2.  a stylewriter may assume their presence
\usepackage{geometry}
%% Some aspects of the preamble are conditional,
%% the LaTeX engine is one such determinant
\usepackage{ifthen}
%% etoolbox has a variety of modern conveniences
\usepackage{etoolbox}
\usepackage{ifxetex,ifluatex}
%% Raster graphics inclusion
\usepackage{graphicx}
%% Color support, xcolor package
%% Always loaded, for: add/delete text, author tools
%% Here, since tcolorbox loads tikz, and tikz loads xcolor
\PassOptionsToPackage{usenames,dvipsnames,svgnames,table}{xcolor}
\usepackage{xcolor}
%% begin: defined colors, via xcolor package, for styling
%% end: defined colors, via xcolor package, for styling
%% Colored boxes, and much more, though mostly styling
%% skins library provides "enhanced" skin, employing tikzpicture
%% boxes may be configured as "breakable" or "unbreakable"
%% "raster" controls grids of boxes, aka side-by-side
\usepackage{tcolorbox}
\tcbuselibrary{skins}
\tcbuselibrary{breakable}
\tcbuselibrary{raster}
%% We load some "stock" tcolorbox styles that we use a lot
%% Placement here is provisional, there will be some color work also
%% First, black on white, no border, transparent, but no assumption about titles
\tcbset{ bwminimalstyle/.style={size=minimal, boxrule=-0.3pt, frame empty,
colback=white, colbacktitle=white, coltitle=black, opacityfill=0.0} }
%% Second, bold title, run-in to text/paragraph/heading
%% Space afterwards will be controlled by environment,
%% independent of constructions of the tcb title
%% Places \blocktitlefont onto many block titles
\tcbset{ runintitlestyle/.style={fonttitle=\blocktitlefont\upshape\bfseries, attach title to upper} }
%% Spacing prior to each exercise, anywhere
\tcbset{ exercisespacingstyle/.style={before skip={1.5ex plus 0.5ex}} }
%% Spacing prior to each block
\tcbset{ blockspacingstyle/.style={before skip={2.0ex plus 0.5ex}} }
%% xparse allows the construction of more robust commands,
%% this is a necessity for isolating styling and behavior
%% The tcolorbox library of the same name loads the base library
\tcbuselibrary{xparse}
%% Hyperref should be here, but likes to be loaded late
%%
%% Inline math delimiters, \(, \), need to be robust
%% 2016-01-31:  latexrelease.sty  supersedes  fixltx2e.sty
%% If  latexrelease.sty  exists, bugfix is in kernel
%% If not, bugfix is in  fixltx2e.sty
%% See:  https://tug.org/TUGboat/tb36-3/tb114ltnews22.pdf
%% and read "Fewer fragile commands" in distribution's  latexchanges.pdf
\IfFileExists{latexrelease.sty}{}{\usepackage{fixltx2e}}
%% Text height identically 9 inches, text width varies on point size
%% See Bringhurst 2.1.1 on measure for recommendations
%% 75 characters per line (count spaces, punctuation) is target
%% which is the upper limit of Bringhurst's recommendations
\geometry{letterpaper,total={340pt,9.0in}}
%% Custom Page Layout Adjustments (use latex.geometry)
%% This LaTeX file may be compiled with pdflatex, xelatex, or lualatex executables
%% LuaTeX is not explicitly supported, but we do accept additions from knowledgeable users
%% The conditional below provides  pdflatex  specific configuration last
%% begin: engine-specific capabilities
\ifthenelse{\boolean{xetex} \or \boolean{luatex}}{%
%% begin: xelatex and lualatex-specific default configuration
\ifxetex\usepackage{xltxtra}\fi
%% realscripts is the only part of xltxtra relevant to lualatex 
\ifluatex\usepackage{realscripts}\fi
%% end:   xelatex and lualatex-specific default configuration
}{
%% begin: pdflatex-specific default configuration
%% We assume a PreTeXt XML source file may have Unicode characters
%% and so we ask LaTeX to parse a UTF-8 encoded file
%% This may work well for accented characters in Western language,
%% but not with Greek, Asian languages, etc.
%% When this is not good enough, switch to the  xelatex  engine
%% where Unicode is better supported (encouraged, even)
\usepackage[utf8]{inputenc}
%% end: pdflatex-specific default configuration
}
%% end:   engine-specific capabilities
%%
%% Fonts.  Conditional on LaTex engine employed.
%% Default Text Font: The Latin Modern fonts are
%% "enhanced versions of the [original TeX] Computer Modern fonts."
%% We use them as the default text font for PreTeXt output.
%% Automatic Font Control
%% Portions of a document, are, or may, be affected by defined commands
%% These are perhaps more flexible when using  xelatex  rather than  pdflatex
%% The following definitions are meant to be re-defined in a style, using \renewcommand
%% They are scoped when employed (in a TeX group), and so should not be defined with an argument
\newcommand{\divisionfont}{\relax}
\newcommand{\blocktitlefont}{\relax}
\newcommand{\contentsfont}{\relax}
\newcommand{\pagefont}{\relax}
\newcommand{\tabularfont}{\relax}
\newcommand{\xreffont}{\relax}
\newcommand{\titlepagefont}{\relax}
%%
\ifthenelse{\boolean{xetex} \or \boolean{luatex}}{%
%% begin: font setup and configuration for use with xelatex
%% Generally, xelatex is necessary for non-Western fonts
%% fontspec package provides extensive control of system fonts,
%% meaning *.otf (OpenType), and apparently *.ttf (TrueType)
%% that live *outside* your TeX/MF tree, and are controlled by your *system*
%% (it is possible that a TeX distribution will place fonts in a system location)
%%
%% The fontspec package is the best vehicle for using different fonts in  xelatex
%% So we load it always, no matter what a publisher or style might want
%%
\usepackage{fontspec}
%%
%% begin: xelatex main font ("font-xelatex-main" template)
%% Latin Modern Roman is the default font for xelatex and so is loaded with a TU encoding
%% *in the format* so we can't touch it, only perhaps adjust it later
%% in one of two ways (then known by NFSS names such as "lmr")
%% (1) via NFSS with font family names such as "lmr" and "lmss"
%% (2) via fontspec with commands like \setmainfont{Latin Modern Roman}
%% The latter requires the font to be known at the system-level by its font name,
%% but will give access to OTF font features through optional arguments
%% https://tex.stackexchange.com/questions/470008/
%% where-and-how-does-fontspec-sty-specify-the-default-font-latin-modern-roman
%% http://tex.stackexchange.com/questions/115321
%% /how-to-optimize-latin-modern-font-with-xelatex
%%
%% end:   xelatex main font ("font-xelatex-main" template)
%% begin: xelatex mono font ("font-xelatex-mono" template)
%% (conditional on non-trivial uses being present in source)
%% end:   xelatex mono font ("font-xelatex-mono" template)
%% begin: xelatex font adjustments ("font-xelatex-style" template)
%% end:   xelatex font adjustments ("font-xelatex-style" template)
%%
%% Extensive support for other languages
\usepackage{polyglossia}
%% Set main/default language based on pretext/@xml:lang value
%% document language code is "en-US", US English
%% usmax variant has extra hypenation
\setmainlanguage[variant=usmax]{english}
%% Enable secondary languages based on discovery of @xml:lang values
%% Enable fonts/scripts based on discovery of @xml:lang values
%% Western languages should be ably covered by Latin Modern Roman
%% end:   font setup and configuration for use with xelatex
}{%
%% begin: font setup and configuration for use with pdflatex
%% begin: pdflatex main font ("font-pdflatex-main" template)
\usepackage{lmodern}
\usepackage[T1]{fontenc}
%% end:   pdflatex main font ("font-pdflatex-main" template)
%% begin: pdflatex mono font ("font-pdflatex-mono" template)
%% (conditional on non-trivial uses being present in source)
%% end:   pdflatex mono font ("font-pdflatex-mono" template)
%% begin: pdflatex font adjustments ("font-pdflatex-style" template)
%% end:   pdflatex font adjustments ("font-pdflatex-style" template)
%% end:   font setup and configuration for use with pdflatex
}
%% Symbols, align environment, commutative diagrams, bracket-matrix
\usepackage{amsmath}
\usepackage{amscd}
\usepackage{amssymb}
%% allow page breaks within display mathematics anywhere
%% level 4 is maximally permissive
%% this is exactly the opposite of AMSmath package philosophy
%% there are per-display, and per-equation options to control this
%% split, aligned, gathered, and alignedat are not affected
\allowdisplaybreaks[4]
%% allow more columns to a matrix
%% can make this even bigger by overriding with  latex.preamble.late  processing option
\setcounter{MaxMatrixCols}{30}
%%
%%
%% Division Titles, and Page Headers/Footers
%% titlesec package, loading "titleps" package cooperatively
%% See code comments about the necessity and purpose of "explicit" option.
%% The "newparttoc" option causes a consistent entry for parts in the ToC 
%% file, but it is only effective if there is a \titleformat for \part.
%% "pagestyles" loads the  titleps  package cooperatively.
\usepackage[explicit, newparttoc, pagestyles]{titlesec}
%% The companion titletoc package for the ToC.
\usepackage{titletoc}
%% begin: customizations of page styles via the modal "titleps-style" template
%% Designed to use commands from the LaTeX "titleps" package
\pagestyle{plain}
%% end: customizations of page styles via the modal "titleps-style" template
%%
%% Create globally-available macros to be provided for style writers
%% These are redefined for each occurence of each division
\newcommand{\divisionnameptx}{\relax}%
\newcommand{\titleptx}{\relax}%
\newcommand{\subtitleptx}{\relax}%
\newcommand{\shortitleptx}{\relax}%
\newcommand{\authorsptx}{\relax}%
\newcommand{\epigraphptx}{\relax}%
%% Create environments for possible occurences of each division
%%
%% Styles for six traditional LaTeX divisions
\titleformat{\part}[display]
{\divisionfont\Huge\bfseries\centering}{\divisionnameptx\space\thepart}{30pt}{\Huge#1}
[{\Large\centering\authorsptx}]
\titleformat{\chapter}[display]
{\divisionfont\huge\bfseries}{\divisionnameptx\space\thechapter}{20pt}{\Huge#1}
[{\Large\authorsptx}]
\titleformat{name=\chapter,numberless}[display]
{\divisionfont\huge\bfseries}{}{0pt}{#1}
[{\Large\authorsptx}]
\titlespacing*{\chapter}{0pt}{50pt}{40pt}
\titleformat{\section}[hang]
{\divisionfont\Large\bfseries}{\thesection}{1ex}{#1}
[{\large\authorsptx}]
\titleformat{name=\section,numberless}[block]
{\divisionfont\Large\bfseries}{}{0pt}{#1}
[{\large\authorsptx}]
\titlespacing*{\section}{0pt}{3.5ex plus 1ex minus .2ex}{2.3ex plus .2ex}
\titleformat{\subsection}[hang]
{\divisionfont\large\bfseries}{\thesubsection}{1ex}{#1}
[{\normalsize\authorsptx}]
\titleformat{name=\subsection,numberless}[block]
{\divisionfont\large\bfseries}{}{0pt}{#1}
[{\normalsize\authorsptx}]
\titlespacing*{\subsection}{0pt}{3.25ex plus 1ex minus .2ex}{1.5ex plus .2ex}
\titleformat{\subsubsection}[hang]
{\divisionfont\normalsize\bfseries}{\thesubsubsection}{1em}{#1}
[{\small\authorsptx}]
\titleformat{name=\subsubsection,numberless}[block]
{\divisionfont\normalsize\bfseries}{}{0pt}{#1}
[{\normalsize\authorsptx}]
\titlespacing*{\subsubsection}{0pt}{3.25ex plus 1ex minus .2ex}{1.5ex plus .2ex}
\titleformat{\paragraph}[hang]
{\divisionfont\normalsize\bfseries}{\theparagraph}{1em}{#1}
[{\small\authorsptx}]
\titleformat{name=\paragraph,numberless}[block]
{\divisionfont\normalsize\bfseries}{}{0pt}{#1}
[{\normalsize\authorsptx}]
\titlespacing*{\paragraph}{0pt}{3.25ex plus 1ex minus .2ex}{1.5em}
%%
%% Styles for five traditional LaTeX divisions
\titlecontents{part}%
[0pt]{\contentsmargin{0em}\addvspace{1pc}\contentsfont\bfseries}%
{\Large\thecontentslabel\enspace}{\Large}%
{}%
[\addvspace{.5pc}]%
\titlecontents{chapter}%
[0pt]{\contentsmargin{0em}\addvspace{1pc}\contentsfont\bfseries}%
{\large\thecontentslabel\enspace}{\large}%
{\hfill\bfseries\thecontentspage}%
[\addvspace{.5pc}]%
\dottedcontents{section}[3.8em]{\contentsfont}{2.3em}{1pc}%
\dottedcontents{subsection}[6.1em]{\contentsfont}{3.2em}{1pc}%
\dottedcontents{subsubsection}[9.3em]{\contentsfont}{4.3em}{1pc}%
%%
%% Begin: Semantic Macros
%% To preserve meaning in a LaTeX file
%%
%% \mono macro for content of "c", "cd", "tag", etc elements
%% Also used automatically in other constructions
%% Simply an alias for \texttt
%% Always defined, even if there is no need, or if a specific tt font is not loaded
\newcommand{\mono}[1]{\texttt{#1}}
%%
%% Following semantic macros are only defined here if their
%% use is required only in this specific document
%%
%% End: Semantic Macros
%% Division Numbering: Chapters, Sections, Subsections, etc
%% Division numbers may be turned off at some level ("depth")
%% A section *always* has depth 1, contrary to us counting from the document root
%% The latex default is 3.  If a larger number is present here, then
%% removing this command may make some cross-references ambiguous
%% The precursor variable $numbering-maxlevel is checked for consistency in the common XSL file
\setcounter{secnumdepth}{0}
%%
%% AMS "proof" environment is no longer used, but we leave previously
%% implemented \qedhere in place, should the LaTeX be recycled
\newcommand{\qedhere}{\relax}
%%
%% A faux tcolorbox whose only purpose is to provide common numbering
%% facilities for most blocks (possibly not projects, 2D displays)
%% Controlled by  numbering.theorems.level  processing parameter
\newtcolorbox[auto counter]{block}{}
%%
%% This document is set to number PROJECT-LIKE on a separate numbering scheme
%% So, a faux tcolorbox whose only purpose is to provide this numbering
%% Controlled by  numbering.projects.level  processing parameter
\newtcolorbox[auto counter]{project-distinct}{}
%% A faux tcolorbox whose only purpose is to provide common numbering
%% facilities for 2D displays which are subnumbered as part of a "sidebyside"
\makeatletter
\newtcolorbox[auto counter, number within=tcb@cnt@block, number freestyle={\noexpand\thetcb@cnt@block(\noexpand\alph{\tcbcounter})}]{subdisplay}{}
\makeatother
%% Localize LaTeX supplied names (possibly none)
%% More flexible list management, esp. for references
%% But also for specifying labels (i.e. custom order) on nested lists
\usepackage{enumitem}
%% hyperref driver does not need to be specified, it will be detected
%% Footnote marks in tcolorbox have broken linking under
%% hyperref, so it is necessary to turn off all linking
%% It *must* be given as a package option, not with \hypersetup
\usepackage[hyperfootnotes=false]{hyperref}
%% Hyperlinking active in electronic PDFs, all links solid and blue
\hypersetup{colorlinks=true,linkcolor=blue,citecolor=blue,filecolor=blue,urlcolor=blue}
\hypersetup{pdftitle={}}
%% If you manually remove hyperref, leave in this next command
\providecommand\phantomsection{}
%% If tikz has been loaded, replace ampersand with \amp macro
%% extpfeil package for certain extensible arrows,
%% as also provided by MathJax extension of the same name
%% NB: this package loads mtools, which loads calc, which redefines
%%     \setlength, so it can be removed if it seems to be in the 
%%     way and your math does not use:
%%     
%%     \xtwoheadrightarrow, \xtwoheadleftarrow, \xmapsto, \xlongequal, \xtofrom
%%     
%%     we have had to be extra careful with variable thickness
%%     lines in tables, and so also load this package late
\usepackage{extpfeil}
%% Custom Preamble Entries, late (use latex.preamble.late)
%% Begin: Author-provided packages
%% (From  docinfo/latex-preamble/package  elements)
%% End: Author-provided packages
%% Begin: Author-provided macros
%% (From  docinfo/macros  element)
%% Plus three from MBX for XML characters
\newcommand{\identity}{\mathrm{id}}
\newcommand{\notdivide}{\nmid}
\newcommand{\notsubset}{\not\subset}
\newcommand{\lcm}{\operatorname{lcm}}
\newcommand{\gf}{\operatorname{GF}}
\newcommand{\inn}{\operatorname{Inn}}
\newcommand{\aut}{\operatorname{Aut}}
\newcommand{\Hom}{\operatorname{Hom}}
\newcommand{\cis}{\operatorname{cis}}
\newcommand{\chr}{\operatorname{char}}
\newcommand{\Null}{\operatorname{Null}}
\newcommand{\transpose}{\text{t}}
\newcommand{\lt}{<}
\newcommand{\gt}{>}
\newcommand{\amp}{&}
%% End: Author-provided macros
%% Title page information for article
\title{}
\date{}
\begin{document}
%% Target for xref to top-level element is document start
\hypertarget{g:article:idp1}{}
%
\begin{enumerate}
\item{}\(\mathcal S\)%
\item{}\(3 + 56 - 13 + 8/2 \)%
\item{}\(2 + 3 = 5\)%
\item{}\(2x = 6\)%
\item{}\(ax^2 + bx + c = 0\)%
\item{}\(a \neq 0\)%
\item{}\(x = \frac{-b \pm \sqrt{b^2 - 4ac}}{2a}\text{.}\)%
\item{}\(2 \cdot 4\)%
\item{}\(p\)%
\item{}\(q\)%
\item{}\(r\)%
\item{}\(s\)%
\item{}\(A\)%
\item{}\(X\)%
\item{}\(a \in A\)%
\item{}\(X = \{ x_1, x_2, \ldots, x_n \}\)%
\item{}\(X = \{ x :x \text{ satisfies }{\mathcal P}\}\)%
\item{}\(E\)%
\item{}\(E = \{2, 4, 6, \ldots \} \quad \text{or} \quad E = \{ x : x \text{ is an even integer and } x \gt 0 \}\text{.}\)%
\item{}\(-3 \notin E\)%
\item{}\(B\)%
\item{}\(A \subset B\)%
\item{}\(B \supset A\)%
\item{}\({\mathbb N} \subset {\mathbb Z} \subset {\mathbb Q} \subset {\mathbb R} \subset {\mathbb C}\text{.}\)%
\item{}\(A \notsubset B\)%
\item{}\(\emptyset\)%
\item{}\(A \cup B\)%
\item{}\(A \cup B = \{x : x \in A \text{ or } x \in B \};\)%
\item{}\(A \cap B = \{x :  x \in A \text{ and } x \in B \}\text{.}\)%
\item{}\(\bigcup_{i = 1}^{n} A_{i} = A_{1} \cup \ldots \cup A_n\)%
\item{}\(\bigcap_{i = 1}^{n} A_{i} = A_{1} \cap \ldots \cap A_n\)%
\item{}\(O\)%
\item{}\(U\)%
\item{}\(A'\)%
\item{}\(A \setminus B = A \cap B'  = \{ x : x \in A \text{ and } x \notin B \}\text{.}\)%
\item{}\(A = \{ x \in {\mathbb R} : 0 \lt x \leq 3 \} \quad \text{and} \quad B = \{ x \in {\mathbb R} : 2 \leq x \lt 4 \}\text{.}\)%
\item{}\(C\)%
\item{}\(A \cup (B \cup C) = (A \cup B) \cup C\)%
\item{}\(A \times B\)%
\item{}\(A = \{ x, y \}\)%
\item{}\(A_1 \times \cdots \times A_n = \{ (a_1, \ldots, a_n): a_i \in A_i \text{ for } i = 1, \ldots, n \}\text{.}\)%
\item{}\(f \subset A \times B\)%
\item{}\(f:A \rightarrow B\)%
\item{}\(f : a \mapsto b\)%
\item{}\(g\)%
\item{}\((g \circ f)(x) = g(f(x))\)%
\item{}\(A =
\begin{pmatrix}
a & b \\
c & d
\end{pmatrix}\text{,}\)%
\item{}\(T_A : {\mathbb R}^2 \rightarrow {\mathbb R}^2\)%
\item{}\({\mathbb R}^m\)%
\item{}\(S = \{ 1,2,3  \}\)%
\item{}\(\pi :S\rightarrow S\)%
\item{}\(h : C \rightarrow D\)%
\item{}\(f(x) = \ln x\)%
\item{}\(f^{-1}(x) = e^x\)%
\item{}\(R \subset X \times X\)%
\item{}\((y, z) \in R\)%
\item{}\(x \sim y\)%
\item{}\(\equiv\)%
\item{}\(\cong\)%
\item{}\(r/s \sim t/u\)%
\item{}\(P\)%
\item{}\(I\)%
\item{}\(Q\)%
\item{}\(X_i \cap X_j = \emptyset\)%
\item{}\(\bigcup_k X_k = X\)%
\item{}\([x] = \{ y \in X : y \sim x \}\)%
\item{}\(r \equiv s \pmod{n}\)%
\item{}\(l\)%
\item{}\(f(x) = \sin x\)%
\item{}\(f: X \rightarrow Y\)%
\item{}\(x \geq y\)%
\item{}\(|x - y| \leq 4\)%
\item{}\(\lambda\)%
\item{}\({\mathbb P}({\mathbb R}) \)%
\item{}\(300!\)%
\item{}\((a + b)^n = \sum_{k = 0}^{n} \binom{n}{k} a^k b^{n - k}\text{,}\)%
\item{}\(a \mid b\)%
\item{}\(d = \gcd(a, b)\)%
\item{}\(\lim_{n \rightarrow \infty} f_n / f_{n + 1} = (\sqrt{5} - 1)/2\)%
\item{}\(\lcm(a,b)\)%
\item{}\(N\)%
\item{}\(\bigtriangleup ABC\)%
\item{}\(\mu_1 \rho_1\)%
\item{}\(\alpha\)%
\item{}\(\beta\)%
\item{}\(\alpha \beta = \identity\)%
\item{}\(G\)%
\item{}\({\mathbb M}_2 ( {\mathbb R})\)%
\item{}\(GL_2({\mathbb R})\)%
\item{}\(\det A = ad - bc \neq 0\)%
\item{}\(I^2 = J^2 = K^2 = -1\)%
\item{}\({\mathbb C}^\ast\)%
\item{}\(g''\)%
\item{}\(g^n = \underbrace{g \cdot g \cdots g}_{n \; \text{times}}\)%
\item{}\(H\)%
\item{}\(\sigma =
\begin{pmatrix}
1 & 2 & \cdots & n \\ a_1 & a_2 & \cdots & a_n
\end{pmatrix}\)%
\item{}\({\mathbb T} = \{ z \in {\mathbb C}^* : |z| =1 \}\)%
\item{}\(\begin{pmatrix}
\cos \theta & -\sin \theta \\
\sin \theta & \cos \theta
\end{pmatrix}\text{,}\)%
\item{}\(H =
\left\{
\begin{pmatrix}
a & b \\
c & d
\end{pmatrix} :
a + d = 0
\right\}\text{.}\)%
\item{}\(Z(G) = \{ x \in G : gx = xg \text{ for all } g \in G \}\)%
\item{}\((d_1, d_2, \ldots, d_k ) \cdot (w_1, w_2, \ldots, w_k ) \equiv 0 \pmod{ n }\)%
\item{}\(\langle a \rangle  = \{ a^k : k \in {\mathbb Z} \}\)%
\item{}\(\overline{z} = a- bi\)%
\item{}\(r \cis \theta\)%
\item{}\(\theta = \arctan \left( \frac{b}{a} \right) = \arctan( - 1) = 315^{\circ}\text{,}\)%
\item{}\(w = s \cis \phi\)%
\item{}\(\begin{array}{c|cccc}
\circ & \identity & \sigma & \tau & \mu \\ \hline
\identity & \identity & \sigma & \tau & \mu \\
\sigma & \sigma & \identity & \mu & \tau \\
\tau & \tau & \mu & \identity & \sigma \\
\mu & \mu & \tau & \sigma & \identity
\end{array}\)%
\item{}\(\sigma( a_i ) = a_{(i \bmod k) + 1}\)%
\item{}\(\sigma_i( x ) = 
\begin{cases}
\sigma( x ) & x \in X_i \\
x   & x \notin X_i
\end{cases}\text{,}\)%
\item{}\({\mathcal O}_{x, \sigma} = \{ y : x \sim y  \}\text{.}\)%
\item{}\({\mathcal L}_H\)%
\item{}\({\mathcal R}_H\)%
\item{}\(p \notdivide a\)%
\item{}\(\gamma\)%
\item{}\({\mathbf p} = 
\begin{pmatrix}
p_1 \\ p_2
\end{pmatrix}\text{.}\)%
\item{}\(f({\mathbf p}) = A {\mathbf p} + {\mathbf b}\text{,}\)%
\item{}\((0.999)^{10,000} \approx 0.00005\text{.}\)%
\item{}\({\mathbf x} = (x_1, \ldots,
x_n)\)%
\item{}\({\mathbf y} = (y_1, \ldots,
y_n)\)%
\item{}\(d_{\min}\)%
\item{}\({\mathbf z} = (00011)\)%
\item{}\({\mathbf c}_1 = (00000)\)%
\item{}\(\Null(H)\)%
\item{}\(\begin{pmatrix}
a_{11} & a_{12} & \cdots & a_{1,n-m} \\
a_{21} & a_{22} & \cdots & a_{2,n-m} \\
\vdots & \vdots & \ddots & \vdots    \\
a_{m1} & a_{m2} & \cdots & a_{m,n-m}
\end{pmatrix}\)%
\item{}\(\begin{array}{cccc}
(000000) & (001101) & (010110) & (011011) \\
(100011) & (101110) & (110101) & (111000).
\end{array}\)%
\item{}\(\delta_{ij} =
\begin{cases}
1 & i = j \\
0 & i \neq j
\end{cases}\)%
\item{}\(H{\mathbf e}_i\)%
\item{}\({\mathbf r}\)%
\item{}\(H =\begin{bmatrix}
0 \amp 1 \amp 0 \amp 1 \amp 0 \\
1 \amp 1 \amp 1 \amp 1 \amp 0 \\
0 \amp 0 \amp 1 \amp 1 \amp 1
\end{bmatrix}\)%
\item{}\(C^\perp = \{ {\mathbf x} \in {\mathbb Z}_2^n :  {\mathbf x} \cdot {\mathbf y} = 0 \text{ for all } {\mathbf y} \in C \}\text{.}\)%
\item{}\(\psi\)%
\item{}\(\prod_{i = 1}^n G_i = G_1 \times G_2 \times \cdots \times G_n\)%
\item{}\(\omega = \cis(2 \pi /n)\)%
\item{}\(\aut(G)\)%
\item{}\(\inn(G)\)%
\item{}\(\Rightarrow\)%
\item{}\(i_g : G \to G\)%
\item{}\(\ker \phi\)%
\item{}\(\eta: G/K \rightarrow \psi(G)\)%
\item{}\(\| {\mathbf x} \| = \sqrt{\langle  {\mathbf x}, {\mathbf x} \rangle}  = \sqrt{x_1^2 + \cdots + x_n^2}\text{.}\)%
\item{}\({\mathbf w}\)%
\item{}\({\mathbf a}_j =
\begin{pmatrix}
a_{1j} \\ a_{2j} \\ \vdots \\ a_{nj}
\end{pmatrix}\)%
\item{}\(\ell\)%
\item{}\(Z_3\rtimes Z_4\)%
\item{}\(v \in {\mathbb R}^2\)%
\item{}\(Y = \{ B, W \}\)%
\item{}\(\widetilde{ \sigma }\)%
\item{}\(q \not\equiv 1 \pmod{p}\)%
\item{}\(1 = 
\begin{pmatrix}
1 & 0 \\
0 & 1
\end{pmatrix},
\quad
{\mathbf i}
=
\begin{pmatrix}
0 & 1 \\
-1 & 0
\end{pmatrix},
\quad
{\mathbf j} =
\begin{pmatrix}
0 & i \\
i & 0
\end{pmatrix},
\quad
{\mathbf k} = 
\begin{pmatrix}
i & 0 \\
0 & -i
\end{pmatrix}\text{,}\)%
\item{}\({\mathbb H}\)%
\item{}\(F =
\left\{
\begin{pmatrix}
1 & 0 \\
0 & 1
\end{pmatrix},
\begin{pmatrix}
1 & 1 \\
1 & 0
\end{pmatrix},
\begin{pmatrix}
0 & 1 \\
1 & 1
\end{pmatrix},
\begin{pmatrix}
0 & 0 \\
0 & 0
\end{pmatrix}
\right\}\)%
\item{}\(\chr R\)%
\item{}\(M\)%
\item{}\(\deg f(x) = n\)%
\item{}\(F\lbrack x \rbrack\)%
\item{}\(\Phi_n(x) = \frac{x^n - 1}{x - 1} = x^{n - 1} + x^{n - 2} + \cdots + x + 1\)%
\item{}\(\deg( p(x) + q(x) ) \leq \max( \deg p(x), \deg q(x) )\)%
\item{}\(\Delta = b^2 - 4ac\)%
\item{}\(\nu : {\mathbb Z}[ \sqrt{3}\, i ] \rightarrow {\mathbb N} \cup \{ 0 \}\)%
\item{}\(a \preceq b\)%
\item{}\(a \vee b\)%
\item{}\(a \wedge b\)%
\item{}\(\succeq\)%
\item{}\(b \not\preceq c\)%
\item{}\(( \Leftarrow)\)%
\item{}\('\)%
\item{}\(V\)%
\item{}\(\dim V =n\)%
\item{}\(S=\{\mathbf{u}\}\)%
\item{}\(W = U \oplus V\)%
\item{}\(\Hom(V, W)\)%
\item{}\(\begin{CD}
E[x]/\langle p(x) \rangle    @>\psi>>  F[x]/\langle q(x) \rangle  \\
@VV{\sigma}V        @VV{\tau}V\\
E(\alpha)     @>\overline{\phi}>>  F(\beta) \\
@VVV        @VVV\\
E     @>\phi>>  F
\end{CD}\)%
\item{}\(\triangle ABC\)%
\item{}\(\gf(p^n)\)%
\item{}\(\begin{CD}
E   @>>>  \{\text{id}\}  \\
@AAA        @VVV\\
L     @>>>  G(E/L) \\
@AAA        @VVV\\
K    @>>>  G(E/K) \\
@AAA        @VVV\\
F    @>>>  G(E/F)
\end{CD}\)%
\item{}\(\zeta\)%
\item{}\(K \supseteq E\)%
\end{enumerate}
%
\end{document}